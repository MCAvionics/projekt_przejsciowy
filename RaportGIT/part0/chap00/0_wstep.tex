Projekt zrealizowano w ramach przedmiotu ,,Projekt przej�ciowy'' na studiach magisterskich na kierunku Automatyka i Robotyka, specjalno�� Robotyka. 

Tematyka projektu dotyczy�a zagadnie� autonomii bezza�ogowego obiektu lataj�cego (ang. \textit{Unmanned Aerial Vehicle - UAV}), powszechnie nazywanego dronem. Urz�dzenie mia�o posta� modelu helikoptera z czterema wirnikami (quadrocoptera). W ramach zaj�� zosta�o poruszonych szereg zagadnie� zwi�zanych z konstrukcj� elektroniki oraz stworzeniem oprogramowania umo�liwiaj�cego zrealizowanie sterowania obiektem, stabilizacj� jego po�o�enia oraz stworzenie autopilota. 

Zadania zosta�y podzielone na problem stabilizacji po�o�enia obiektu w punkcie oraz stabilizacj� w pomieszczeniu. Nast�pnie przetestowano zaimplementowane algorytmy. Testowana platforma zosta�a oparta o sterownik PixHawk, kt�ry zawiera� jednostk� inercyjn� IMU (ang. \textit{inertial measurement unit}), wej�cie dla odbiornika radiowego oraz sterowniki do silnik�w. Jednostka inercyjna zawiera�a �yroskop, akcelerometr i magnetometr. Opr�cz sterownika, quadrocopter posiada� tak�e komputer typu embedded oparty o platform� Atom, kt�ry dokonywa� wszelkich niezb�dnych oblicze� na podstawie pomiar�w z czujnik�w i obrazu z kamer. Komputer ten by� platform� docelow�, na kt�rej zaimplementowano autopilota opartego o system ROS.  