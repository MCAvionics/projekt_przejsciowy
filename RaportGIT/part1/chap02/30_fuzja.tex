
\noindent \textbf{Algorytm stabilizacji wysoko�ci}

\noindent Dane wej�ciowe:
\begin{itemize}
	\item $d_1$, $d_2$ -- pomiary z dw�ch czujnik�w odleg�o�ci: g�rnego i dolnego,
	\item $q$ -- orientacja quadrocoptera wzgl�dem zewn�trznego uk�adu odniesienia, w postaci kwaternionu (z Pixhawka),
\end{itemize}

\noindent Dane wyj�ciowe:
\begin{itemize}
	\item $F_z$ -- si�a w osi \emph{z} quadrocoptera. 
\end{itemize}
~\\
\textbf{Kroki algorytmu}
\begin{enumerate}
	\item Obliczenie rzeczywistych odleg�o�ci od pod�o�a i sufitu na podstawie odczyt�w sensor�w oraz orientacji quadrocoptera wzgl�dem zewn�trznego uk�adu odniesienia.
	\item Fuzja pomiar�w -- wybranie do stabilizacji odleg�o�ci, kt�ra wolniej si� zmienia (mniejsza pochodna). Celem jest uniezale�nienie wysoko�ci od niewielkich przeszk�d.
	\item Regulator PID:
\begin{itemize}
	\item wej�cie: wybrana wysoko�� $h$,
	\item wyj�cie: sterowanie w postaci si�y w osi Z.
\end{itemize} 
\end{enumerate}