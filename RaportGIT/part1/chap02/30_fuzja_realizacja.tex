Algorytm stabilizacji wysoko�ci zrealizowano w postaci pakietu �rodowiska ROS. Zasada dzia�ania pakietu opiera si� na pobieraniu pomiar�w czujnik�w odleg�o�ci z topiku \textsf{/pomiary} oraz orientacji quadrocoptera ze sterownika Pixhawk za po�rednictwem protoko�u \textsf{mavlink}, pakietu \textsf{mavros} oraz topiku \textsf{/mavros/imu/data}. Wysoko�� robota publikowana jest w topiku \textsf{/height}.

W celu obliczenia rzeczywistej wysoko�ci quadrocoptera, jego orientacja w postaci kwaternionu jest konwertowana do reprezentacji w postaci k�t�w RPY (Roll, Pitch, Yaw), z wykorzystaniem przedstawionej poni�ej funkcji.

\begin{lstlisting}[language=C, frame=single,tabsize=2, caption={Funkcja konwertuj�ca orientacj� w postaci kwaternion�w do k�t�w RPY}] 
void quaternionsToRPY(double q0, double q1, double q2, 
double q3, double *roll, double *pitch, double *yaw)
{
  *roll = atan2(2*(q0*q1+q2*q3), 1-2*(q1*q1+q2*q2))*180/M_PI; 
  *pitch = asin(2*(q0*q2-q3*q1)) * 180 / M_PI;
  *yaw = atan2(2*(q0*q3+q1*q2), 1-2*(q2*q2 + q3*q3))*180/M_PI;
}
\end{lstlisting}
~~\\
Dane wej�ciowe z czujnik�w odleg�o�ci s� wst�pnie filtrowane, a nast�pnie na ich podstawie oraz na podstawie orientacji obliczana jest rzeczywista wysoko��. W dalszej cz�ci przedstawiono spos�b pobierania danych o orientacji z odpowiedniego topika oraz spos�b wyznaczania wysoko�ci quadrocoptera.

\begin{lstlisting}[language=C, frame=single,tabsize=2, caption={Pobieranie danych o orientacji quadrocoptera}] 
void HeightQuadro::sensorsCallback(
			const height_control::Sensors::ConstPtr& msg)
{
	_upMeasurements[_measureIter] = msg->dane[UP_SENSOR_NR];
	_downMeasurements[_measureIter] = msg->dane[DOWN_SENSOR_NR];
	
	if(++_measureIter >= PREFILTER_A)
		_measureIter = 0;
	preFilter();
	heightFromDistance();
} 
\end{lstlisting}

\begin{lstlisting}[language=C, frame=single,tabsize=2, caption={Obliczanie wysoko�ci robota}]
 void HeightQuadro::heightFromDistance()
{	
	double _roll, _pitch, _yaw;
	quaternionsToRPY(_orientation.x, _orientation.y, 
	_orientation.z, _orientation.w, &_roll, &_pitch, &_yaw);
	_realUpHeight 	= _filteredUp[1]/(cos(_roll)*cos(_pitch));		
	_realDownHeight = _filteredDown[1]/(cos(_roll)*cos(_pitch));
}
\end{lstlisting}
