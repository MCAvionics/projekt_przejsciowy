Modu� ROS zawiera wiele ciekawych i przydatnych w sterowaniu robotami funkcji. Pierwszym zadaniem po zainstalowaniu pakietu ROS, by�o uruchomienie kamery internetowej $2-D$, wykorzystuj�c wbudowane funkcje. W tym celu ROS wykorzystuje pakiet Gstream, kt�rego konfiguracja jest opisana poni�ej.

\begin{enumerate}
\item Przygotowanie repozytorium\\
$sudo apt-get install ubuntu-restricted-extras$
\item Instalacja pakietu GStream (w naszym przypadku manualnie, poniewa� z automatem zdarzaj� si� problemy)\\
$sudo apt-get install gstreamer-dbus-media-service gstreamer-tools gstreamer0.10-alsa gstreamer0.10-buzztard$\\
$gstreamer0.10-buzztard-doc gstreamer0.10-crystalhd gstreamer0.10-doc$\\
$gstreamer0.10-ffmpeg gstreamer0.10-ffmpeg-dbg$\\
$gstreamer0.10-fluendo-mp3 gstreamer0.10-gconf gstreamer0.10-gnomevfs$\\
$gstreamer0.10-gnonlin gstreamer0.10-gnonlin-dbg$\\
$gstreamer0.10-gnonlin-doc gstreamer0.10-hplugins gstreamer0.10-nice$\\
$gstreamer0.10-packagekit gstreamer0.10-plugins-bad$\\
$gstreamer0.10-plugins-bad-doc gstreamer0.10-plugins-bad-multiverse$\\
$gstreamer0.10-plugins-base gstreamer0.10-plugins-base-apps$\\
$gstreamer0.10-plugins-base-dbg gstreamer0.10-plugins-base-doc$\\
$gstreamer0.10-plugins-cutter gstreamer0.10-plugins-good$\\
\item Instalacja funkcji $"gscam"$\\
$rosdep install gscam$\\
$rosmake gscam$
\end{enumerate}

Gdy pakiety zosta�y przygotowane i dysponujemy kamerk�, nast�pnym krokiem jest weryfikacja dzia�ania oprogramowania. W terminalu nale�y wpisa� nast�puj�ce polecenia:

\begin{itemize}
\item $cd$ $bin$
\item $export$ $GSCAM\_CONFIG="v4l2src$ \\
$device=/dev/video0! video/x-raw-rgb ! ffmpegcolorspace"$
\item $rosrun$ $gscam$ $gscam$
\end{itemize}

Powy�ej zaprezentowana metoda mo�e nie zadzia�a� w ka�dej dystrybucji Ubuntu i ROS'a. Za ka�dym razem nale�y zweryfikowa� posiadane wersje i si�gn�� do dokumentacji udost�pnionej przez producenta. Je�li jednak wszystko si� uda�o, w okienku wy�wietli si� obraz z kamery, bez dodatkowych informacji i funkcji.