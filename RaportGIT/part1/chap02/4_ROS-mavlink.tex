Modu� MAVLINK zosta� przygotowany przez producent�w tak, aby dzia�a� wsp�lnie z ROS'em. Dlatego te� jego instalacja oraz p�niejsze u�ytkowanie jest bardzo proste. Aby zainstalowa� MAVLINK'a nale�y wykona� nast�puj�ce kroki:

\begin{itemize}
\item Pobranie paczek z np. GitHub?a \\
$git$ $clone$ $https://github.com/mavlink/mavlink\_ros.git$
\item Kompilacja\\
$cd$ $mavlink\_ros$\\
$rosmake$
\end{itemize}

Tak zainstalowany modu� daje pi�kne i intuicyjne �rodowisko do podgl�du tego co dzieje si� "na" i wok� lataj�cego robota. Aby m�c odbiera� dane z czujnik�w, kamer itp., nale�y "podpi��" port pod widget programu.
W tym celu u�ywamy opcji $Network => Add Link$. Dane otrzymane w ten spos�b mo�na wyeksportowa� oraz wizualizowa� r�wnie� w zewn�trznych �rodowiskach, np. w Matlab'ie. Wszystkie widgety mo�na do woli edytowa�(rozmieszczenie, ilo�� itp) poprzez klikni�cie prawym przyciskiem myszy i wybranie w�a�ciwo�ci danego elementu.
