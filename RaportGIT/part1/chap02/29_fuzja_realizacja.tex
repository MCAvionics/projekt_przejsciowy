
Do realizacji fuzji z algorytmu SVO pozyskiwane s� informacje o pozycji kamery. Dzi�ki znanej pozycji pocz�tkowej $[x_0, y_0, z_0]$ kamery mo�na obliczy� przesuni�cie. Na tej podstawie wysy�ana jest wiadomo�� do sterownika o przesuni�ciu, uzyskanym z algorytmu wizualnego. Pozycj� kamery nale�y przekonwertowa� na pozycj� quadrocoptera przez odpowiedni� macierz rotacji (komentarz: jeszcze nie wiem jak�, bo nie wiem, z kt�rej kamery w ko�cu b�d� korzysta� i gdzie ona b�dzie). Wiadmo�� wysy�ana do sterownika jest umieszczana w topicu o typie \textit{geometry\_msg/Vector3}. Parametr $\delta$, kt�ry trzeba dobra� podczas testu, b�dzie decydowa� czy wyst�pi�o przesuni�cie, czy te� nie. Poni�szy wydruk pokazuje funkcj� rosow� \textit{Collback}, kt�ra uzyskuje informacj� o pozycji kamery z topicu \texttt{/svo/points}.
\begin{lstlisting}[language=C++, frame=single,tabsize=1, caption={Uruchomienie kamery}] 
<launch>
  <node name="usb_cam" pkg="usb_cam" type="usb_cam_node"
  // output="screen" >
    <param name="video_device" value="/dev/video0" />
    <param name="image_width" value="640" />
    <param name="image_height" value="480" />
    <param name="pixel_format" value="mjpeg" />
    <param name="camera_frame_id" value="usb_cam" />
    <param name="io_method" value="mmap"/>
  </node>
</launch>
\end{lstlisting}