
Do realizacji fuzji z algorytmu SVO pozyskiwane s� informacje o pozycji kamery. Dzi�ki znanej pozycji pocz�tkowej $[x_0, y_0, z_0]$ kamery mo�na obliczy� przesuni�cie. Na tej podstawie wysy�ana jest wiadomo�� do sterownika o przesuni�ciu, uzyskanym z algorytmu wizualnego. Pozycj� kamery nale�y przekonwertowa� na pozycj� quadrocoptera przez odpowiedni� macierz rotacji (komentarz: jeszcze nie wiem jak�, bo nie wiem, z kt�rej kamery w ko�cu b�d� korzysta� i gdzie ona b�dzie). Wiadomo�� wysy�ana do sterownika jest umieszczana w topicu o typie \textit{geometry\_msg/Vector3}. Parametr $\delta$, kt�ry trzeba dobra� podczas testu, b�dzie decydowa� czy wyst�pi�o przesuni�cie, czy te� nie. Poni�szy wydruk pokazuje funkcj� rosow� \textit{Callback}, kt�ra uzyskuje informacj� o pozycji kamery z topicu \texttt{/svo/points}... CDN
\begin{lstlisting}[language=C++, frame=single,tabsize=1, caption={Pobranie warto�ci po�o�enia kamery}] 
void chatterCallback(const visualization_msgs::Marker::ConstPtr& msg)
{
  t[0][0]=msg->pose.position.x;
  t[0][1]=msg->pose.position.y;
  t[0][2]=msg->pose.position.z;
}
\end{lstlisting}