Debugowanie, jest to proces systematycznego redukowania liczby b��d�w w oprogramowaniu b�d� w systemie mikroporcesorowym, kt�ry polega na kontrolowanym wykonywaniu programu pod nadzorem narz�dzia debugujacego. \\
Proces usuwania b��d�w mo�na podzieli� na kilka etap�w:
\begin{enumerate}
\item Reprodukcja b��du
\item Wyizolowanie �r�d�a b��du
\item Identyfikacja przyczyny awarii
\item Usuni�cie defektu
\item Weryfikacja powodzenia naprawy
\end{enumerate}
Centralnym punktem procesu usuwania b��d�w programu jest obserwacja jego wykonania w celu lokalizacji �r�d�a usterki. Zadanie to u�atwia g��wne narz�dzie do dynamicznej analizy programu, czyli \textit{Debugger}. Umo�liwia on:
\begin{itemize}
\item wykonywanie programu w trybie pracy krokowej lub z zastawianiem tzw. pu�apek
\item podgl�danie i ewentualna zmiana zawarto�ci rejestr�w, pami�ci itd. 
\end{itemize}
W celu zdalnego debugowania wykorzystane zosta�o po��czenie quadrocoptera z GroundStation, a dok�adnie z programem qGroundControl. Pozwala on na podgl�d na zmienne stanu i status coptera w czasie rzeczywistym. \\
Po��czenie sterownika komputera pok�adowego z qGroundControl nast�puje poprzez SSH. Komunikaty o stanie quadrocoptera wysy�ane s� w formie komunikat�w gcc oraz komunikat�w: ROS\_INFO, ROS\_ERROR oraz ROS\_FATAL. 