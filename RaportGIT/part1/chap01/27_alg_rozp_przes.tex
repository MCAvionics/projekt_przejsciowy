Ze wzgl�du na form� zawod�w, quadrocopter b�dzie porusza� si� w
zamkni�tym pomieszczeniu, gdzie jak wiadomo sygna� GPS
zawodzi. Dlatego jednym z~wa�nych zada� niniejszego projektu, by�a detekcja
przesuni�cia, ruchu quadrocoptera, kt�ra powinna by� dokonywana
automatycznie przy wykorzystaniu sekwencji obraz�w z~kamery
wizyjnej.Wizualna odometria mo�e wsp�pracowa� z innymi rozwi�zaniami,
dlatego do stabilizacji w punkcie system wizyjny zostanie po��czony z
czujnikami laserowymi oraz ultrad�wi�kowymi. Zminimalizuje to b��d
estymacji ruchu, a co za tym idzie, b�dzie mo�na przeciwdzia�a�
dryfowi. Poniewa� projekt opiera si� na platformie ROS za�o�ono, �e istniej� ju� gotowe rozwi�zania problemu detekcji ruchu kamery
i~dokonano przegl�du dost�pnych algorytm�w. 
Do implementacji i~bada� wybrano algorytm Lucas--Kanade oraz dwa
algorytmy monoSLAM PTAM i~SVO 
