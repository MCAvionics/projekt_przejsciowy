Platforma na jakiej mieli�my okazj� pracowa� spe�nia�a za�o�enia architektury x86. Wykorzystanie takiego sprz�tu pozwala na zainstalowanie sytemu operacyjnego linux oraz wykorzystanie wielu dost�pnych bibliotek. W projektach, kt�re maj� na celu implementacj� algorytm�w sterowania, na szczeg�ln� uwag� zas�uguj� nast�puj�ce mo�liwo�ci:
\begin{itemize}
    \item proste wykonywanie cyklicznych zada�,
    \item �atwa integracja danych pomiarowych z r�nych czujnik�w.
\end{itemize}
W systemach operacyjnych z rodziny linux cykliczne wykonywanie zada� umo�liwia nam us�uga systemowa o nazwie cron. Pozwala ona za pomoc� wpisu w odpowiedniej tabeli uruchamia� cyklicznie pewien program lub skrypt. W tego typu systemach dost�pne s� r�wnie� pakiety takie jak xenomai oraz orocos, kt�re poza cyklicznym wykonywaniem zada� daj� nam tak�e wsparcie czasu rzeczywistego.
