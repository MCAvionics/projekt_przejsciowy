W celu przeszukiwania pomieszczenia nale�a�o ustali� scenariusz dzia�ania i lotu quadrocoptera. Ustalono, �e najbardziej efektywnym, a zarazem najbezpieczniejszym dla pojazdu scenariuszem przeszukiwania b�dzie lot wzd�u� �ciany. \\
W tym celu, po wleceniu do pomieszczenia, quadrocopter ustawia si� ty�em, r�wnolegle do �ciany i utrzymuje bezpieczn� od niej odleg�o�� wykorzystuj�c sterownik antykolizyjny. Lot odbywa si� w jednym sta�ym kierunku ( w prawo lub w lewo) a� do napotkania przez czujniki znajduj�ce si� na jego bokach nast�pnej �ciany. Wtedy pojazd obraca si� tak, aby ponownie jego ty� znalaz� si� r�wnolegle do nast�pnej �ciany, a ca�y proces powtarza si� a� do znalezienia si� w punkcie pocz�tkowym. \\
Ustalono �e taki scenariusz lotu pozwoli na:
\begin{itemize}
\item uniknianie przeszk�d na drodze,
\item dok�adne zeskanowanie pomieszczenia
\item utworzenie dok�adnej mapy pomieszczenia
\item bezpieczny powr�t do punktu pocz�tkowego po wcze�niejszej trasie.
\end{itemize}